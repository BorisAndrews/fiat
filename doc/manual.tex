\documentclass{fenicsmanual}
\usepackage{graphicx}
\usepackage{amssymb}
\usepackage{amsmath}
\usepackage{url}

\textwidth = 6.5in
\textheight = 9in
\oddsidemargin = 0in
\evensidemargin = 0in
\topmargin = 0.0 in
\headheight = 0.0 in
\headsep = 0.0 in
\parskip = 0.2in
\parindent = 0.0in

\newtheorem{theorem}{Theorem}
\newtheorem{lemma}[theorem]{Lemma}
\newtheorem{corollary}[theorem]{Corollary}
\newtheorem{definition}{Definition}
\newtheorem{remark}{Remark}
\newtheorem{example}{Example}
\newcommand{\tuple}[2]{<\!#1,#2\!>}
\newcommand{\meet}{\wedge}
\newcommand{\bigmeet}{\bigwedge}
\def\proof{\par{\it Proof}. \ignorespaces}

\def\endproof{\vbox{\hrule height0.6pt\hbox{%
   \vrule height1.3ex width0.6pt\hskip0.8ex
   \vrule width0.6pt}\hrule height0.6pt
  }}

\newcommand{\N}{{\sf
    N\hspace*{-1.0ex}\rule{0.15ex}{1.3ex}\hspace*{1.0ex}}}

\title{FIAT 0.2.4 Users' Manual}
\author{Robert C. Kirby}
\begin{document}
\maketitle

\chapter{Introduction}
FIAT (FInite element Automatic Tabulator) is a Python package for
defining and evaluating a wide range of different finite element basis
functions for numerical partial differential equations.  It is
intended to make ``difficult'' elements such as high-order
Brezzi-Douglas-Marini~\cite{} elements usable by providing
abstractions so that they may be implemented succinctly and hence
treated as a black box.  FIAT is intended for use at two different
levels.  For one, it is designed to provide a standard API for finite
element bases so that programmers may use whatever elements they need
in their code.  At a lower level, it provides necessary infrastructure to
rapidly deploy new kinds of finite elements without expensive symbolic
computation or tedious algebraic manipulation.
It is my goal that a large number of people use FIAT without ever
knowing it.  Thanks to several ongoing projects such as
Sundance~\cite{}, FFC~\cite{}, and PETSc~\cite{}, it is becoming
possible to to define finite element methods using mathematical
notation in some high-level or domain-specific language.  The primary
shortcoming of these projects is their lack of support for general
elements.  It is one thing to ``provide hooks'' for general elements,
but absent a tool such as FIAT, these hooks remain mainly empty.  As
these projects mature, I hope to expose users of the finite element
method to the exotic world of potentially high-degree finite element
on unstructured grids using the best elements in $H^1$,
$H(\mathrm{div})$, and $H(\mathrm{curl})$.

In this brief (and still developing) guide, I will first
present the high-level API for users who wish to instantiate a finite
element on a reference domain and evaluate its basis functions and
derivatives at some quadrature points.  Then, I will explain some of
the underlying infrastructure so as to demonstrate how to add new
elements.

\chapter{Installation}
FIAT uses the standard Python \texttt{distutils} tools.  From the top
directory, one executes \texttt{python setup.py install}.  This will
put FIAT into the \texttt{site-packages} directory.  Super-user
permission (such as \texttt{su} or \texttt{sudo}) may be required to
write to this directory.  For more configuration options, one may type
\texttt{python setup.py --help} or consult the online Python
documentation at \texttt{http://docs.python.org/inst/inst.html}

FIAT requires the commonly used \verb.Numeric. package.


\chapter{Using FIAT: A tutorial with Lagrange elements}
\section{Importing FIAT}
FIAT is organized as a package in Python, consisting of several
modules.  In order to get some of the packages, we use the line
\begin{verbatim}
from FIAT import Lagrange, quadrature, shapes
\end{verbatim}
This loads several modules for the Lagrange elements, quadrature
rules, and the simplicial element shapes which FIAT implements.  The
roles each of these plays will become clear shortly.

\section{Important note}
Throughout, FIAT defines the reference elements based on the interval
$(-1,1)$ rather than the more common $(0,1)$.  So, the one-dimensional
reference element is $(-1,1)$, the three vertices of the reference
triangle are $(-1,-1),(1,-1),(1,-1)$, and the four vertices of the
reference tetrahedron are $(-1,-1,-1),(1,-1,-1),(-1,1,-1),(-1,-1,1)$.

\section{Instantiating elements}
FIAT uses a lightweight object-oriented infrastructure to define
finite elements.  The \verb.Lagrange. module contains a class
\verb.Lagrange. modeling the Lagrange finite element family.  This
class is a subclass of some \verb.FiniteElement. class contained in
another module (\verb.polynomial. to be precise).  So, having imported
the \verb.Lagrange. module, we can create the Lagrange element of
degree \verb.2. on triangles by
\begin{verbatim}
shape = shapes.TRIANGLE
degree = 2
U = Lagrange.Lagrange( shape , degree )
\end{verbatim}
Here, \verb*shapes.TRIANGLE* is an integer code indicating the two
dimensional simplex.  \verb.shapes. also defines
\verb.LINE. and \verb.TETRAHEDRON..  Most of the
upper-level interface to FIAT is dimensionally abstracted over element
shape.

The class \verb.FiniteElement. supports three methods, modeled on the
abstract definition of Ciarlet.  These methods are
\verb.domain_shape()., \verb.function_space()., and \verb.dual_basis()..
The first of these returns the code for the shape and the second
returns the nodes of the finite element (including information related
to topological association of nodes with mesh entities, needed for
creating degree of freedom orderings).  

\section{Quadrature rules}
FIAT implements arbitrary-order collapsed quadrature, as discussed in
Karniadakis and Sherwin~\cite{}, for the simplex of dimension one,
two, or three.  The simplest way to get a quadrature rule is through
the function \verb.make_quadrature(shape,m)., which takes a shape code
and an integer indicating the number of points per direction.  For
building element matrices using quadratics, we will typically need a
second or third order integration rule, so we can get such a rule by
\begin{verbatim}
>>> Q = quadrature.make_quadrature( shape , 2 )
\end{verbatim}
This uses two points in each direction on the reference square, then
maps them to the reference triangle.  We may get a
\verb/Numeric.array/ of the quadrature weights with the method
\verb/Q.get_weights()/ and a list of tuples storing the quadrature
points with the method \verb/Q.get_points()/.

\section{Tabulation}
FIAT provides functions for tabulating the element basis functions and
their derivatives.  To get the \verb.FunctionSpace. object, we do
\begin{verbatim}
>>> Ufs = U.function_space()
\end{verbatim}
To get the values of each basis function at each of the quadrature
points, we use the \verb.tabulate(). method
\begin{verbatim}
>>> Ufs.tabulate( Q.get_points() )
array([[ 0.22176167, -0.12319761, -0.11479229, -0.06377178],
       [-0.11479229, -0.06377178,  0.22176167, -0.12319761],
       [-0.10696938,  0.18696938, -0.10696938,  0.18696938],
       [ 0.11074286,  0.19356495,  0.41329796,  0.72239423],
       [ 0.41329796,  0.72239423,  0.11074286,  0.19356495],
       [ 0.47595918,  0.08404082,  0.47595918,  0.08404082]])
\end{verbatim}
This returns a two-dimensional \verb/Numeric.array/ with rows for each
basis function and columns for each input point.

Also, finite element codes require tabulation of the basis functions'
derivatives.  Each \verb/FunctionSpace/ object also provides a method
\verb/tabulate_jet(i,xs)/ that returns a list of Python dictionaries.
The \verb.i.th entry of the list is a dictionary storing the values of
all \verb.i.th order derivatives.  Each dictionary maps a multiindex
(a tuple of length \verb.i.) to the table of the associated partial
derivatives of the basis functions at those points.  For example,
\begin{verbatim}
>>> Ufs_jet = Ufs.tabulate_jet( 1 , Q.get_points() )
\end{verbatim}
tabulates the zeroth and first partial derivatives of the function
space at the quadrature points.  Then,
\begin{verbatim}
>>> Ufs_jet[0]
{(0, 0): array([[ 0.22176167, -0.12319761, -0.11479229, -0.06377178],
       [-0.11479229, -0.06377178,  0.22176167, -0.12319761],
       [-0.10696938,  0.18696938, -0.10696938,  0.18696938],
       [ 0.11074286,  0.19356495,  0.41329796,  0.72239423],
       [ 0.41329796,  0.72239423,  0.11074286,  0.19356495],
       [ 0.47595918,  0.08404082,  0.47595918,  0.08404082]])}
\end{verbatim}
gives us a dictionary mapping the only zeroth-order partial derivative
to the values of the basis functions at the quadrature points.  More
interestingly, we may get the first derivatives in the x- and y-
directions with
\begin{verbatim}
>>> Ufs_jet[1][(1,0)]
array([[-0.83278049, -0.06003983,  0.14288254,  0.34993778],
       [-0.14288254, -0.34993778,  0.83278049,  0.06003983],
       [ 0.        ,  0.        ,  0.        ,  0.        ],
       [ 0.31010205,  1.28989795,  0.31010205,  1.28989795],
       [-0.31010205, -1.28989795, -0.31010205, -1.28989795],
       [ 0.97566304,  0.40997761, -0.97566304, -0.40997761]])
>>> Ufs_jet[1][(0,1)] 
array([[ -8.32780492e-01,  -6.00398310e-02,   1.42882543e-01,   3.49937780e-01],
       [  7.39494156e-17,   4.29608279e-17,   4.38075188e-17,   7.47961065e-17],
       [ -1.89897949e-01,   7.89897949e-01,  -1.89897949e-01,   7.89897949e-01],
       [  3.57117457e-01,   1.50062220e-01,   1.33278049e+00,   5.60039831e-01],
       [  1.02267844e+00,  -7.29858118e-01,   4.70154051e-02,  -1.13983573e+00],
       [ -3.57117457e-01,  -1.50062220e-01,  -1.33278049e+00,  -5.60039831e-01]])
\end{verbatim}

\chapter{Lower-level API}
Not only does FIAT provide a high-level library interface for users to
evaluate existing finite element bases, but it also provides
lower-level tools.  Here, we survey these tools module-by-module.

\section{shapes.py}
FIAT currenly only supports simplicial reference elements, but does so
in a fairly dimensionally-independent way (up to tetrahedra).

\section{jacobi.py}
This is a low-level module that tabulates the Jacobi polynomials and
their derivatives, and also provides Gauss-Jacobi points.  This module
will seldom if ever be imported directly by users.  For more
information, consult the documentation strings and source code.

\section{expansions.py}


\section{quadrature.py}


\section{polynomial.py}



\chapter{Wish list and open problems}
While FIAT is highly functional as a tool for tabulating basis
functions at quadrature points, there are a lot of interesting
things to do.  In case anybody wants to help out, I have
chosen to describe some of these issues here.

\section{Stable/fast VDM inversion}

\section{Symmetric quadrature rules}

\section{Declarative top-level language}

\section{Integration with SMART-type tools}


\end{document}
