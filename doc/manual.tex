\documentclass[12pt]{article}
\usepackage{graphicx}
\usepackage{amssymb}
\usepackage{amsmath}
\usepackage{url}

\textwidth = 6.5in
\textheight = 9in
\oddsidemargin = 0in
\evensidemargin = 0in
\topmargin = 0.0 in
\headheight = 0.0 in
\headsep = 0.0 in
\parskip = 0.2in
\parindent = 0.0in

\newtheorem{theorem}{Theorem}
\newtheorem{lemma}[theorem]{Lemma}
\newtheorem{corollary}[theorem]{Corollary}
\newtheorem{definition}{Definition}
\newtheorem{remark}{Remark}
\newtheorem{example}{Example}
\newcommand{\tuple}[2]{<\!#1,#2\!>}
\newcommand{\meet}{\wedge}
\newcommand{\bigmeet}{\bigwedge}
\def\proof{\par{\it Proof}. \ignorespaces}

\def\endproof{\vbox{\hrule height0.6pt\hbox{%
   \vrule height1.3ex width0.6pt\hskip0.8ex
   \vrule width0.6pt}\hrule height0.6pt
  }}

\newcommand{\R}{{\sf R\hspace*{-0.9ex}\rule{0.15ex}{1.5ex}\hspace*{0.9ex}}}
\newcommand{\N}{{\sf
    N\hspace*{-1.0ex}\rule{0.15ex}{1.3ex}\hspace*{1.0ex}}}

\title{FIAT 0.1.0 Users' Manual}
\author{Robert C. Kirby}
\begin{document}
\maketitle

\section{Introduction}
FIAT (FInite element Automatic Tabulator) is a Python package for
defining and evaluating a wide range of different finite element basis
functions for numerical partial differential equations.  It is
intended to make ``difficult'' elements such as high-order
Brezzi-Douglas-Marini~\cite{} elements usable by providing
abstractions so that they may be implemented succinctly and hence
treated as a black box.  FIAT is intended for use at two different
levels.  For one, it is designed to provide a standard API for finite
element bases so that programmers may use whatever elements they need
in their code.  At a lower level, it provides necessary infrastructure to
rapidly deploy new kinds of finite elements without expensive symbolic
computation or tedious algebraic manipulation.
It is my goal that a large number of people use FIAT without ever
knowing it.  Thanks to several ongoing projects such as
Sundance~\cite{}, FFC~\cite{}, and PETSc~\cite{}, it is becoming
possible to to define finite element methods using mathematical
notation in some high-level or domain-specific language.  The primary
shortcoming of these projects is their lack of support for general
elements.  It is one thing to ``provide hooks'' for general elements,
but absent a tool such as FIAT, these hooks remain mainly empty.  As
these projects mature, I hope to expose users of the finite element
method to the exotic world of potentially high-degree finite element
on unstructured grids using the best elements in $H^1$,
$H(\mathrm{div})$, and $H(\mathrm{curl})$.

In this brief (and still developing) guide, I will first
present the high-level API for users who wish to instantiate a finite
element on a reference domain and evaluate its basis functions and
derivatives at some quadrature points.  Then, I will explain some of
the underlying infrastructure so as to demonstrate how to add new
elements.

\section{Using FIAT: A tutorial with Lagrange elements}
\subsection{Importing FIAT}
FIAT is organized as a package in Python, consisting of several
modules.  In order to get some of the packages, we use the line
\begin{verbatim}
from FIAT import Lagrange, quadrature, shapes
\end{verbatim}
This loads several modules for the Lagrange elements, quadrature
rules, and the simplicial element shapes which FIAT implements.  The
roles each of these plays will become clear shortly.
\subsection{Instantiating elements}
FIAT uses a lightweight object-oriented infrastructure to define
finite elements.  The \verb.Lagrange. module contains a class
\verb.Lagrange. modeling the Lagrange finite element family.  This
class is a subclass of some \verb.FiniteElement. class contained in
another module (\verb.polynomial. to be precise).  So, having imported
the \verb.Lagrange. module, we can create the Lagrange element of
degree \verb.2. on triangles by
\begin{verbatim}
shape = shapes.TRIANGLE
degree = 2
U = Lagrange.Lagrange( shape , degree )
\end{verbatim}
Here, \verb*shapes.TRIANGLE* is an integer code indicating the two
dimensional simplex.  \verb.shapes. also defines
\verb.LINE. and \verb.TETRAHEDRON..  Most of the
upper-level interface to FIAT is dimensionally abstracted over element
shape.

The class \verb.FiniteElement. supports three methods, modeled on the
abstract definition of Ciarlet.  These methods are
\verb.domain_shape()., \verb.function_space()., and \verb.dual_basis()..
The first of these returns the code for the shape and the second
returns the nodes of the finite element (including information related
to topological association of nodes with mesh entities, needed for
creating degree of freedom orderings).  

\subsection{Quadrature rules}
\subsection{Tabulation}
Tabulation of function values; Tabulation of derivatives, jets.

\end{document}
